\documentclass[uplatex]{jsarticle}
\title{\vspace{-30mm}\flushleft{\huge{第6回\ (\ \ 月\ \ 日)\ Unit14〜15}+$\alpha$}}
\author{\Large{\hspace{90mm}名前\hspace{40mm}得点\hspace{20mm}点}}
\usepackage[top=20mm, bottom=10mm, left=15mm, right=15mm]{geometry}
\parindent=0pt
\usepackage{setspace}
\renewcommand{\baselinestretch}{1.2}
\date{\empty}
\pagestyle{empty}
\usepackage[dvipdfmx]{graphicx}
\usepackage{listings}
\usepackage{jlisting}

\lstset{
  basicstyle={\ttfamily},
  identifierstyle={\small},
  commentstyle={\smallitshape},
  keywordstyle={\small\bfseries},
  ndkeywordstyle={\small},
  stringstyle={\small\ttfamily},
  frame={tb},
  breaklines=true,
  columns=[l]{fullflexible},
  numbers=left,
  xrightmargin=0zw,
  xleftmargin=3zw,
  numberstyle={\scriptsize},
  stepnumber=1,
  numbersep=1zw,
  lineskip=-0.5ex
}

\begin{document}
\maketitle
\vspace{-10mm}
\Large 問1.英単語の意味を書きなさい。\\
\large\ (1) heart\hspace{\fill}(1)\underline{\hspace{35mm}}\\
\ (2) then\hspace{\fill}(2)\underline{\hspace{35mm}}\\
\ (3) war\hspace{\fill}(3)\underline{\hspace{35mm}}\\
\ (4) way\hspace{\fill}(4)\underline{\hspace{35mm}}\\
\ (5) understand\hspace{\fill}(5)\underline{\hspace{35mm}}\\
\ (6) still\hspace{\fill}(6)\underline{\hspace{35mm}}\\
\ (7) job\hspace{\fill}(7)\underline{\hspace{35mm}}\\
\Large 問2.日本語の意味を表す英単語を書きなさい。\\
\large\ (8) 〜に答える\hspace{\fill}(8)\underline{\hspace{35mm}}\\
\ (9) 〜が聞こえる\hspace{\fill}(9)\underline{\hspace{35mm}}\\
\ (10) 大きさ\hspace{\fill}(10)\underline{\hspace{35mm}}\\
\ (11) 日付\hspace{\fill}(11)\underline{\hspace{35mm}}\\
\ (12) 〜を習得する\hspace{\fill}(12)\underline{\hspace{35mm}}\\
\ (13) 場所\hspace{\fill}(13)\underline{\hspace{35mm}}\\
\ (14) 〜を意味する\hspace{\fill}(14)\underline{\hspace{35mm}}\\
\Large 問3.空所に入る英語を番号で答えなさい。\\
\large\ (15) 優しい人\\
\hspace{10mm}a kind (\hspace{10mm})\\
\hspace{10mm}\textcircled{\normalsize1}person \textcircled{\normalsize2}people 
\textcircled{\normalsize3}human \textcircled{\normalsize4}being 
\hspace{\fill}(15)\underline{\hspace{35mm}}\\
\ (16) テレビを見る\\
\hspace{10mm}(\hspace{10mm}) television\\
\hspace{10mm}\textcircled{\normalsize1}watch \textcircled{\normalsize2}see 
\textcircled{\normalsize3}look \textcircled{\normalsize4}find 
\hspace{\fill}(16)\underline{\hspace{35mm}}\\
\ (17) 警察を呼ぶ\\
\hspace{10mm}call the (\hspace{10mm})\\
\hspace{10mm}\textcircled{\normalsize1}police \textcircled{\normalsize2}politician 
\textcircled{\normalsize3}professor \textcircled{\normalsize4}physician 
\hspace{\fill}(17)\underline{\hspace{35mm}}\\
\ (18) 私に本を見せる\\
\hspace{10mm}(\hspace{10mm}) me a book\\
\hspace{10mm}\textcircled{\normalsize1}display \textcircled{\normalsize2}reveal 
\textcircled{\normalsize3}introduce \textcircled{\normalsize4}show 
\hspace{\fill}(18)\underline{\hspace{35mm}}\\
\ (19) 多くの命\\
\hspace{10mm}many (\hspace{10mm})\\
\hspace{10mm}\textcircled{\normalsize1}live \textcircled{\normalsize2}life 
\textcircled{\normalsize3}heart \textcircled{\normalsize4}lives 
\hspace{\fill}(19)\underline{\hspace{35mm}}\\
\ (20) 私の名前を知っている\\
\hspace{10mm}(\hspace{10mm}) my name\\
\hspace{10mm}\textcircled{\normalsize1}recognize \textcircled{\normalsize2}know 
\textcircled{\normalsize3}notice \textcircled{\normalsize4}realize 
\hspace{\fill}(20)\underline{\hspace{35mm}}\\
  % \begin{lstlisting}[caption=Main.java, numbers=left, frame=single, breaklines=true]
% \lstinputlisting[caption=Main.java, numbers=left, frame=single, breaklines=true]{./newlang3/Main.java}
\end{document}
