\documentclass[uplatex]{jsarticle}

\title{\vspace{-30mm}\flushleft{\huge{第4回\ (\ \ 月\ \ 日)\ Unit8〜10}}}
\author{\Large{\hspace{90mm}名前\hspace{40mm}得点\hspace{20mm}点}}
\usepackage[top=20mm, bottom=10mm, left=15mm, right=15mm]{geometry}
\parindent=0pt
\usepackage{setspace}
\renewcommand{\baselinestretch}{1.2}
\date{\empty}
\pagestyle{empty}
\usepackage[dvipdfmx]{graphicx}
\usepackage{listings}
\usepackage{jlisting}

\lstset{
  basicstyle={\ttfamily},
  identifierstyle={\small},
  commentstyle={\smallitshape},
  keywordstyle={\small\bfseries},
  ndkeywordstyle={\small},
  stringstyle={\small\ttfamily},
  frame={tb},
  breaklines=true,
  columns=[l]{fullflexible},
  numbers=left,
  xrightmargin=0zw,
  xleftmargin=3zw,
  numberstyle={\scriptsize},
  stepnumber=1,
  numbersep=1zw,
  lineskip=-0.5ex
}

\begin{document}
\maketitle
\vspace{-10mm}
\Large 問1.英単語の意味を書きなさい。\\
\large\ (1) land\hspace{10mm}\hspace{\fill}(1)\underline{\hspace{35mm}}\\
\ (2) join\hspace{\fill}(2)\underline{\hspace{35mm}}\\
\ (3) stand\hspace{\fill}(3)\underline{\hspace{35mm}}\\
\ (4) jump\hspace{\fill}(4)\underline{\hspace{35mm}}\\
\ (5) fly\hspace{\fill}(5)\underline{\hspace{35mm}}\\
\ (6) plant\hspace{\fill}(6)\underline{\hspace{35mm}}\\
\ (7) meet\hspace{\fill}(7)\underline{\hspace{35mm}}\\
\Large 問2.日本語の意味を表す英単語を書きなさい。\\
\large\ (8) 祭り\hspace{\fill}(8)\underline{\hspace{35mm}}\\
\ (9) 〜を歓迎する\hspace{\fill}(9)\underline{\hspace{35mm}}\\
\ (10) 地球\hspace{\fill}(10)\underline{\hspace{35mm}}\\
\ (11) 動く\hspace{\fill}(11)\underline{\hspace{35mm}}\\
\ (12) (〜を)飲む\hspace{\fill}(12)\underline{\hspace{35mm}}\\
\ (13) 光\hspace{\fill}(13)\underline{\hspace{35mm}}\\
\ (14) 店\hspace{\fill}(14)\underline{\hspace{35mm}}\\
\Large 問3.空所に入る英語を番号で答えなさい。\\
\large\ (15) 英語を話す\\
\hspace{10mm}(\hspace{10mm}) English\\
\hspace{10mm}\textcircled{\normalsize1}speak \textcircled{\normalsize2}tell 
\textcircled{\normalsize3}say \textcircled{\normalsize4}talk 
\hspace{\fill}(15)\underline{\hspace{35mm}}\\
\ (16) カレーライスを作る\\
\hspace{10mm}(\hspace{10mm}) curry and rice\\
\hspace{10mm}\textcircled{\normalsize1}boil \textcircled{\normalsize2}cook 
\textcircled{\normalsize3}bake \textcircled{\normalsize4}steam 
\hspace{\fill}(16)\underline{\hspace{35mm}}\\
\ (17) 母は私に一生懸命に勉強しなくてはいけないといった。\\
\hspace{10mm}My mother (\hspace{10mm}) me that I had to study hard.\\
\hspace{10mm}\textcircled{\normalsize1}said \textcircled{\normalsize2}told 
\textcircled{\normalsize3}talked \textcircled{\normalsize4}saw 
\hspace{\fill}(17)\underline{\hspace{35mm}}\\
\ (18) 新鮮な空気\\
\hspace{10mm}fresh (\hspace{10mm})\\
\hspace{10mm}\textcircled{\normalsize1}atmosphere \textcircled{\normalsize2}air 
\textcircled{\normalsize3}wind \textcircled{\normalsize4}aim 
\hspace{\fill}(18)\underline{\hspace{35mm}}\\
\ (19) いすに座る\\
\hspace{10mm}(\hspace{10mm}) on a chair\\
\hspace{10mm}\textcircled{\normalsize1}sit \textcircled{\normalsize2}set 
\textcircled{\normalsize3}seat \textcircled{\normalsize4}sick 
\hspace{\fill}(19)\underline{\hspace{35mm}}\\
\ (20) 火をおこす\\
\hspace{10mm}make a (\hspace{10mm})\\
\hspace{10mm}\textcircled{\normalsize1}light \textcircled{\normalsize2}flame 
\textcircled{\normalsize3}fire \textcircled{\normalsize4}torch 
\hspace{\fill}(20)\underline{\hspace{35mm}}\\
  % \begin{lstlisting}[caption=Main.java, numbers=left, frame=single, breaklines=true]
% \lstinputlisting[caption=Main.java, numbers=left, frame=single, breaklines=true]{./newlang3/Main.java}
\end{document}
