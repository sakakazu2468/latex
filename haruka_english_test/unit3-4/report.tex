\documentclass[uplatex]{jsarticle}

\title{\vspace{-30mm}\flushleft{\huge{第2回\ (\ \ 月\ \ 日)\ Unit3〜4}}}
\author{\Large{\hspace{90mm}名前\hspace{40mm}得点\hspace{20mm}点}}
\usepackage[top=20mm, bottom=10mm, left=15mm, right=15mm]{geometry}
\parindent=0pt
\usepackage{setspace}
\renewcommand{\baselinestretch}{1.2}
\date{\empty}
\pagestyle{empty}
\usepackage[dvipdfmx]{graphicx}
\usepackage{listings}
\usepackage{jlisting}

\lstset{
  basicstyle={\ttfamily},
  identifierstyle={\small},
  commentstyle={\smallitshape},
  keywordstyle={\small\bfseries},
  ndkeywordstyle={\small},
  stringstyle={\small\ttfamily},
  frame={tb},
  breaklines=true,
  columns=[l]{fullflexible},
  numbers=left,
  xrightmargin=0zw,
  xleftmargin=3zw,
  numberstyle={\scriptsize},
  stepnumber=1,
  numbersep=1zw,
  lineskip=-0.5ex
}

\begin{document}
\maketitle
\vspace{-10mm}
\Large 問1.英単語の意味を書きなさい。\\
\large\ (1) short\hspace{10mm}\hspace{\fill}(1)\underline{\hspace{35mm}}\\
\ (2) member\hspace{\fill}(2)\underline{\hspace{35mm}}\\
\ (3) math\hspace{\fill}(3)\underline{\hspace{35mm}}\\
\ (4) question\hspace{\fill}(4)\underline{\hspace{35mm}}\\
\ (5) test\hspace{\fill}(5)\underline{\hspace{35mm}}\\
\ (6) low\hspace{\fill}(6)\underline{\hspace{35mm}}\\
\ (7) clean\hspace{\fill}(7)\underline{\hspace{35mm}}\\
\Large 問2.日本語の意味を表す英単語を書きなさい。\\
\large\ (8) レッスン\hspace{\fill}(8)\underline{\hspace{35mm}}\\
\ (9) 知らせ\hspace{\fill}(9)\underline{\hspace{35mm}}\\
\ (10) 親切な\hspace{\fill}(10)\underline{\hspace{35mm}}\\
\ (11) 言語\hspace{\fill}(11)\underline{\hspace{35mm}}\\
\ (12) 考え\hspace{\fill}(12)\underline{\hspace{35mm}}\\
\ (13) 集団\hspace{\fill}(13)\underline{\hspace{35mm}}\\
\ (14) 科学\hspace{\fill}(14)\underline{\hspace{35mm}}\\
\Large 問3.空所に入る英語を番号で答えなさい。\\
\large\ (15) 小さい町\\
\hspace{10mm}a (\hspace{10mm}) town\\
\hspace{10mm}\textcircled{\normalsize1}tiny \textcircled{\normalsize2}little \textcircled{\normalsize3}low \textcircled{\normalsize4}small 
\hspace{\fill}(15)\underline{\hspace{35mm}}\\
\ (16) 高い山\\
\hspace{10mm}a (\hspace{10mm}) mountain\\
\hspace{10mm}\textcircled{\normalsize1}tall \textcircled{\normalsize2}high \textcircled{\normalsize3}large \textcircled{\normalsize4}big 
\hspace{\fill}(16)\underline{\hspace{35mm}}\\
\ (17) 長い川\\
\hspace{10mm}a (\hspace{10mm}) river\\
\hspace{10mm}\textcircled{\normalsize1}load \textcircled{\normalsize2}long \textcircled{\normalsize3}road \textcircled{\normalsize4}rest 
\hspace{\fill}(17)\underline{\hspace{35mm}}\\
\ (18) 背の高い少年\\
\hspace{10mm}a (\hspace{10mm}) boy\\
\hspace{10mm}\textcircled{\normalsize1}long \textcircled{\normalsize2}tall \textcircled{\normalsize3}big \textcircled{\normalsize4}call 
\hspace{\fill}(18)\underline{\hspace{35mm}}\\
\ (19) 大きい家\\
\hspace{10mm}a (\hspace{10mm}) house\\
\hspace{10mm}\textcircled{\normalsize1}big \textcircled{\normalsize2}begin \textcircled{\normalsize3}bit \textcircled{\normalsize4}large 
\hspace{\fill}(19)\underline{\hspace{35mm}}\\
\ (20) 偉大な歌手\\
\hspace{10mm}a (\hspace{10mm}) singer\\
\hspace{10mm}\textcircled{\normalsize1}good \textcircled{\normalsize2}greet \textcircled{\normalsize3}great \textcircled{\normalsize4}legend 
\hspace{\fill}(20)\underline{\hspace{35mm}}\\
  % \begin{lstlisting}[caption=Main.java, numbers=left, frame=single, breaklines=true]
% \lstinputlisting[caption=Main.java, numbers=left, frame=single, breaklines=true]{./newlang3/Main.java}
\end{document}
