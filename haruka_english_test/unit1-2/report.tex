\documentclass[uplatex]{jsarticle}

\title{\vspace{-30mm}\flushleft{\huge{第1回\ (\ \ 月\ \ 日)\ Unit1〜2}}}
\author{\Large{\hspace{90mm}名前\hspace{40mm}得点\hspace{20mm}点}}
\usepackage[top=20mm, bottom=10mm, left=15mm, right=15mm]{geometry}
\parindent=0pt
\usepackage{setspace}
\renewcommand{\baselinestretch}{1.2}
\date{\empty}
\pagestyle{empty}
\usepackage[dvipdfmx]{graphicx}
\usepackage{listings}
\usepackage{jlisting}

\lstset{
  basicstyle={\ttfamily},
  identifierstyle={\small},
  commentstyle={\smallitshape},
  keywordstyle={\small\bfseries},
  ndkeywordstyle={\small},
  stringstyle={\small\ttfamily},
  frame={tb},
  breaklines=true,
  columns=[l]{fullflexible},
  numbers=left,
  xrightmargin=0zw,
  xleftmargin=3zw,
  numberstyle={\scriptsize},
  stepnumber=1,
  numbersep=1zw,
  lineskip=-0.5ex
}

\begin{document}
\maketitle
\vspace{-10mm}
\Large 問1.英単語の意味を書きなさい。\\
\large\ (1) bring a flower\hspace{10mm}\hspace{\fill}(1)\underline{花を\hspace{27mm}}\\
\ (2) go\hspace{\fill}(2)\underline{\hspace{35mm}}\\
\ (3) have a comic book\hspace{\fill}(3)\underline{漫画本を\hspace{19mm}}\\
\ (4) look at A\hspace{\fill}(4)\underline{\hspace{35mm}}\\
\ (5) go 〜ing\hspace{\fill}(5)\underline{\hspace{35mm}}\\
\ (6) keep a picture\hspace{\fill}(6)\underline{写真を\hspace{23mm}}\\
\ (7) take an umbrella\hspace{\fill}(7)\underline{傘を\hspace{27mm}}\\
\Large 問2.日本語の意味を表す英単語を書きなさい。\\
\large\ (8) Aに〜するように頼む\hspace{\fill}(8)\underline{\hspace{35mm}}\\
\ (9) 〜を食べる\hspace{\fill}(9)\underline{ h\hspace{32mm}}\\
\ (10) 〜を作る\hspace{\fill}(10)\underline{\hspace{35mm}}\\
\ (11) (やって)来る\hspace{\fill}(11)\underline{\hspace{35mm}}\\
\ (12) 〜を手に入れる\hspace{\fill}(12)\underline{\hspace{35mm}}\\
\ (13) 〜に見える\hspace{\fill}(13)\underline{\hspace{35mm}}\\
\ (14) (時間)がかかる\hspace{\fill}(14)\underline{\hspace{35mm}}\\
\Large 問3.空所に入る英語を番号で答えなさい。\\
\large\ (15) 私を幸せにする\\
\hspace{10mm}(\hspace{10mm}) me happy\\
\hspace{10mm}\textcircled{\normalsize1}create \textcircled{\normalsize2}make \textcircled{\normalsize3}take \textcircled{\normalsize4}lead 
\hspace{\fill}(15)\underline{\hspace{35mm}}\\
\ (16) 駅に着く\\
\hspace{10mm}(\hspace{10mm}) to the station\\
\hspace{10mm}\textcircled{\normalsize1}come \textcircled{\normalsize2}go \textcircled{\normalsize3}get \textcircled{\normalsize4}arrive 
\hspace{\fill}(15)\underline{\hspace{35mm}}\\
\ (17) 平和をもたらす\\
\hspace{10mm}(\hspace{10mm}) peace\\
\hspace{10mm}\textcircled{\normalsize1}take \textcircled{\normalsize2}have \textcircled{\normalsize3}come \textcircled{\normalsize4}bring 
\hspace{\fill}(15)\underline{\hspace{35mm}}\\
\ (18) 私にいつ...かを尋ねる\\
\hspace{10mm}(\hspace{10mm}) me when ...\\
\hspace{10mm}\textcircled{\normalsize1}ask \textcircled{\normalsize2}speak \textcircled{\normalsize3}hear \textcircled{\normalsize4}have 
\hspace{\fill}(15)\underline{\hspace{35mm}}\\
\ (19) 部屋をきれいにしておく\\
\hspace{10mm}(\hspace{10mm}) a room clean\\
\hspace{10mm}\textcircled{\normalsize1}hold \textcircled{\normalsize2}keep \textcircled{\normalsize3}make \textcircled{\normalsize4}let 
\hspace{\fill}(15)\underline{\hspace{35mm}}\\
\ (20) あなたのパーティーに行く\\
\hspace{10mm}(\hspace{10mm}) to your party\\
\hspace{10mm}\textcircled{\normalsize1}come \textcircled{\normalsize2}look \textcircled{\normalsize3}ask \textcircled{\normalsize4}excite 
\hspace{\fill}(15)\underline{\hspace{35mm}}\\
  % \begin{lstlisting}[caption=Main.java, numbers=left, frame=single, breaklines=true]
% \lstinputlisting[caption=Main.java, numbers=left, frame=single, breaklines=true]{./newlang3/Main.java}
\end{document}
