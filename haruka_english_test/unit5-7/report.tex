\documentclass[uplatex]{jsarticle}

\title{\vspace{-30mm}\flushleft{\huge{第3回\ (\ \ 月\ \ 日)\ Unit5〜7}}}
\author{\Large{\hspace{90mm}名前\hspace{40mm}得点\hspace{20mm}点}}
\usepackage[top=20mm, bottom=10mm, left=15mm, right=15mm]{geometry}
\parindent=0pt
\usepackage{setspace}
\renewcommand{\baselinestretch}{1.2}
\date{\empty}
\pagestyle{empty}
\usepackage[dvipdfmx]{graphicx}
\usepackage{listings}
\usepackage{jlisting}

\lstset{
  basicstyle={\ttfamily},
  identifierstyle={\small},
  commentstyle={\smallitshape},
  keywordstyle={\small\bfseries},
  ndkeywordstyle={\small},
  stringstyle={\small\ttfamily},
  frame={tb},
  breaklines=true,
  columns=[l]{fullflexible},
  numbers=left,
  xrightmargin=0zw,
  xleftmargin=3zw,
  numberstyle={\scriptsize},
  stepnumber=1,
  numbersep=1zw,
  lineskip=-0.5ex
}

\begin{document}
\maketitle
\vspace{-10mm}
\Large 問1.英単語の意味を書きなさい。\\
\large\ (1) village\hspace{10mm}\hspace{\fill}(1)\underline{\hspace{35mm}}\\
\ (2) fast\hspace{\fill}(2)\underline{\hspace{35mm}}\\
\ (3) enjoy\hspace{\fill}(3)\underline{\hspace{35mm}}\\
\ (4) vacation\hspace{\fill}(4)\underline{\hspace{35mm}}\\
\ (5) farm\hspace{\fill}(5)\underline{\hspace{35mm}}\\
\ (6) trip\hspace{\fill}(6)\underline{\hspace{35mm}}\\
\ (7) smile\hspace{\fill}(7)\underline{\hspace{35mm}}\\
\Large 問2.日本語の意味を表す英単語を書きなさい。\\
\large\ (8) 弱い\hspace{\fill}(8)\underline{\hspace{35mm}}\\
\ (9) 電話\hspace{\fill}(9)\underline{\hspace{35mm}}\\
\ (10) (時間・時期が)早い\hspace{\fill}(10)\underline{\hspace{35mm}}\\
\ (11) 座席\hspace{\fill}(11)\underline{\hspace{35mm}}\\
\ (12) 〜に感じる\hspace{\fill}(12)\underline{\hspace{35mm}}\\
\ (13) 難しい\hspace{\fill}(13)\underline{\hspace{35mm}}\\
\ (14) (時間・時期が)遅れた\hspace{\fill}(14)\underline{\hspace{35mm}}\\
\Large 問3.空所に入る英語を番号で答えなさい。\\
\large\ (15) 強い体\\
\hspace{10mm}a (\hspace{10mm}) body\\
\hspace{10mm}\textcircled{\normalsize1}tough \textcircled{\normalsize2}strong 
\textcircled{\normalsize3}powerful \textcircled{\normalsize4}muscular 
\hspace{\fill}(15)\underline{\hspace{35mm}}\\
\ (16) ミルクが欲しくて泣く\\
\hspace{10mm}(\hspace{10mm}) for milk\\
\hspace{10mm}\textcircled{\normalsize1}want \textcircled{\normalsize2}shout 
\textcircled{\normalsize3}tear \textcircled{\normalsize4}cry 
\hspace{\fill}(16)\underline{\hspace{35mm}}\\
\ (17) 水を与える\\
\hspace{10mm}(\hspace{10mm}) water\\
\hspace{10mm}\textcircled{\normalsize1}take \textcircled{\normalsize2}give 
\textcircled{\normalsize3}feed \textcircled{\normalsize4}contribute 
\hspace{\fill}(17)\underline{\hspace{35mm}}\\
\ (18) 君のものと異なる\\
\hspace{10mm}be (\hspace{10mm}) from yours\\
\hspace{10mm}\textcircled{\normalsize1}various \textcircled{\normalsize2}individual 
\textcircled{\normalsize3}different \textcircled{\normalsize4}difference 
\hspace{\fill}(18)\underline{\hspace{35mm}}\\
\ (19) 楽しみを持つ\\
\hspace{10mm}have (\hspace{10mm})\\
\hspace{10mm}\textcircled{\normalsize1}fun \textcircled{\normalsize2}enjoyment 
\textcircled{\normalsize3}interest \textcircled{\normalsize4}pleasure 
\hspace{\fill}(19)\underline{\hspace{35mm}}\\
\ (20) 私のと同じかばん\\
\hspace{10mm}(\hspace{10mm}) (\hspace{10mm}) bag as mine\\
\hspace{10mm}\textcircled{\normalsize1}a same \textcircled{\normalsize2}a one 
\textcircled{\normalsize3}the same \textcircled{\normalsize4}the one 
\hspace{\fill}(20)\underline{\hspace{35mm}}\\
  % \begin{lstlisting}[caption=Main.java, numbers=left, frame=single, breaklines=true]
% \lstinputlisting[caption=Main.java, numbers=left, frame=single, breaklines=true]{./newlang3/Main.java}
\end{document}
